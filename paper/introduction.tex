\section{Introduction}
The act of simplifying a sentence in a more understandable version of itself is called ``simplification".
It is quite an easy task for us humans to perform, but how could a computer possibly know how to
find simpler synonyms and substitute them into a target phrase? \\ 
The current state-of-the-art approach relies on edit operations to mark editable parts in the 
target sentence thanks to an \textit{encoder-decoder} structure \cite{dong-etal-2019-editnts}.
However, it doesn't seem that in literature a good-performing method relying on \textit{auto-regressive transformers},
with no regards for editing operations, exists. 
Basically all existing simplifiers rely on the edit operations paradigm. Why is that? \\
In this paper we proposed an approach relying on different models of \textit{auto-regressive transformers} to validate
the fact that they are not fit to perform as simplifiers when compared to their supervised counterparts.
